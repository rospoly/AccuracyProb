\documentclass[10pt,conference]{IEEEtran}
\usepackage[utf8]{inputenc}
\usepackage{amsmath}
\usepackage{amsfonts}
\usepackage{amssymb}
\usepackage{graphicx}
\usepackage{cleveref}
\usepackage{relsize}

%% EDITING
\newcommand{\ie}{i.e.\ }
\newcommand{\eg}{e.g.\ }
\crefname{section}{\S}{\S\S}
\crefname{subsection}{\S}{\S\S}
\crefname{subsubsection}{\S}{\S\S}
\crefname{equation}{}{}

%% MATHS
\newcommand{\round}{\mathrm{Round}}
\newcommand{\pfp}{+_{\mathrm{fp}}}
\newcommand{\absv}[1]{\vert #1\vert}
\newcommand{\Pro}{\mathbb{P}}
\newcommand{\F}{\mathbb{F}}
\newcommand{\R}{\mathbb{R}}
\newcommand{\ceil}[1]{\lceil #1 \rceil}
\newcommand{\floor}[1]{\lfloor #1 \rfloor}
\newcommand{\intvl}[1]{\mathlarger{\left[\right.}  #1 \mathlarger{\left.\right]}}
\newcommand{\inv}{^{-1}}
\newcommand{\fintvl}[1][x]{\mathlarger{\lfloor}#1,#1\mathlarger{\rceil}}
% Machine operation
\newcommand{\mop}{\mathtt{op_m}}
% Infinite-precision operation
\newcommand{\iop}{\mathrm{op}}


\title{A probabilistic approach to the accuracy and stability of numerical algorithms}
\author{George Constantinides and Fredrik Dahlqvist \\ Department of Electrical and Electronic Engineering\\ Imperial College London}
\begin{document}
\maketitle

\begin{abstract}

\end{abstract}

\section{Introduction}

IEEE arithmetic \cite{ieee754} is traditionally modelled mathematically as follows \cite{higham2002accuracy}: if $x,y$ are two floating-point representable numbers and $\iop\in\{+,-,\times,\div\}$ is an infinite-precision arithmetic operation, then the floating-point precision implementation $\mop$ of $\iop$ must satisfy:
\begin{align}
x~\mop~y=(x~\iop~y)(1+\delta), \qquad\absv{\delta}\leq u\label{eq:traditional}
\end{align}
where $u$ us the unit roundoff for the given precision. \cref{eq:traditional} says that the machine implementation of an arithmetic operation can make a \emph{relative error} of size $\delta$ \emph{for some} $\delta\in\left[-u,u\right]$. The `\emph{for some}' is essential: this is a \emph{non-deterministic model}, we have no control whatsoever over which $\delta$ appears in \cref{eq:traditional}. This means that numerical analysis based on this model \emph{must} consider \emph{all} possible values $\delta$, \ie numerical analysis based on \cref{eq:traditional} is fundamentally a \emph{worst-case analysis}. 

It follows from the perspective of \cref{eq:traditional} that any program doing arithmetic is in fact a non-deterministic program. Moreover, since the output of such a program might very well turn out to be the input of another program doing arithmetic, one should also consider non-deterministic inputs. This is in fact what happens in practice with tools for numerical analysis like Daisy \cite{darulova2018daisy} or FPTaylor \cite{solovyev2018rigorous} which require for each variable of the program  a range of possible values in order to perform a worst-case analysis.

For a wide variety of programs however, it makes sense to assume that the inputs are \emph{probabilistic} rather than non-deterministic; that is to say we have some statistical model of the inputs of the program. This situation is in fact incredibly common. The inputs of one numerical routine are frequently generated randomly by another numerical routine, for example in a gradient descent optimization, or in a Bayesian inference algorithm. Similarly, sensors on a cyber-physical system can feed analog signals which are very well modelled statistically, to a numerical program processing these signals. 

If the inputs of a program have a known distribution, then it becomes possible, at least in principle, to ask the question: \textit{how likely are the inputs generating the worst-case rounding errors generated by the model \cref{eq:traditional}?} Typically, these inputs will occur very infrequently, and in this this respect the non-deterministic model can be overly pessimistic since worst-case behaviours might in practice be such rare events that they are never encountered. 

In this paper we will explore a quantitative model which formally looks very similar to \cref{eq:traditional}, namely
\begin{align}
x~\mop~y=(x~\iop~y)(1+\delta), \qquad\delta\sim dist \label{eq:probabilistic}
\end{align}
but now $\delta$ is \emph{sampled} from $dist$, a probability distribution whose support is $\left[-u,u\right]$. In other words we move from a non-deterministic model of rounding errors to a \emph{probabilistic} model of rounding errors. This model will allow us to formalise and answer questions like \textit{What is the average rounding error?} \textit{What is the worst-case error with $99.9\%$ accuracy?}

The probabilistic model \cref{eq:probabilistic} is not new. ...



\begin{itemize}
\item Recent probabilistic approaches to numerical accuracy (Nick Higham, Ilse Ipsen) focus on large/high-dimensional problems and rely on concentration of measure inequalities which provide useful information in this context.
\item Our approach is exact (no concentration of measure inequalities) but tailored to smaller programs. 
\item Test-cases: `small' scalar products for accuracy and ray tracing algorithm (Slabs method) for stability. 
\end{itemize} 

\section{A probabilistic model of rounding errors}

\subsection{Rounding error distribution}\label{subsec:error_dist}

\begin{itemize}
\item Compute the distribution of the random variable $\frac{X-\round(X)}{X}$ given the random variable $X$
\item Show that under mild assumptions on the distribution of $X$, the distribution of rounding errors is given by the roughly trapezoidal distribution of Fig \ref{fig:trapeze}.
\begin{figure}[ht!]
\includegraphics[scale=0.55]{trapeze_dist}
\caption{Typical distribution of rounding errors (in unit roundoffs)}
\label{fig:trapeze}
\end{figure}
\end{itemize}


\subsubsection{Derivation of the rounding error distribution}
Suppose a continuous (real-valued) random variable $X$ is distributed according to a probability density function $f$ (for example $X$ could model a stochastic, analog input signal), we can explicitly compute the distribution of the relative error induced by the quantization procedure (for example when the analog signal is digitized). We will `normalize' the result by working in units of $u$, the unit roundoff. Since the relative rounding error lies in the interval $[-u,u]$, the normalized relative error will lie in the interval $[-1,1]$.  We start by looking at the cumulative distribution function, \ie we compute the function:
\begin{align*}
c(t):=\Pro\left(\frac{X-\round(X)}{X}\leq tu\right)
\end{align*}
where $u$ is the unit roundoff. Clearly $\round(X)$ can take any value in $\F$, and given $x\in\F$ it is only possible to have $\round(X)=x$ if $X\in \left[\floor{x},\ceil{x}\right[$. This partitions $\R$

\subsection{Probabilistic version of the IEEE 754 standard}\label{subsec:prob_ieee754}

\begin{itemize}
\item Replace the usual non-deterministic 
\[
x\pfp y=(x+y)(1+\varepsilon), \absv{\varepsilon}\leq u
\]
with
\[
x\pfp y=(x+y)(1+\varepsilon)
\]
where $\varepsilon$ is a random variable of known distribution.
\item The `typical' distribution of $\varepsilon$ is described in \cref{subsec:error_dist}.
\end{itemize}

\section{Rounding error distribution of simple programs}

\subsection{Probabilistic programs}
\begin{itemize}
\item What they are: (1) programs which can sample from known probability distributions, (2) Programs whose inputs can be probabilistic.
\item The probabilistic model of IEEE 754 of \cref{subsec:prob_ieee754} turns any deterministic program into a probabilistic one.
\end{itemize}

\subsection{How probabilistic programs process probabilistic inputs}
\begin{itemize}
\item Pushing a distribution through a deterministic function
\item Pushing a distribution through a probabilistic function
\item Pushing a distribution through an \texttt{if then else} statement
\item Pushing a distribution through a simple program
\item Application to programs with probabilistic rounding errors (\cref{subsec:prob_ieee754})
\end{itemize}

\section{Probabilistic accuracy and stability}

\subsection{Accuracy: scalar products}

\begin{itemize}
\item Probabilistic accuracy
\item Comparison with classical worst-case analysis (\cite[3.1]{higham2002accuracy})
\end{itemize}

\subsection{Stability: ray tracing via the slabs method}

\bibliographystyle{plain}
\bibliography{bib/constantinides-dahlqvist}

\end{document}