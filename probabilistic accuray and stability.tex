\documentclass[10pt,a4paper]{article}
\usepackage[utf8]{inputenc}
\usepackage{amsmath}
\usepackage{amsthm}
\usepackage{amsfonts}
\usepackage{amssymb}
\usepackage{relsize}
\usepackage{nicefrac}
\usepackage{bbm}
\usepackage{color}
\usepackage[all]{xy}
\usepackage{listings}
\lstset{
	language=C,
	basicstyle=\ttfamily,
	keywordstyle=\color{blue},
	morekeywords={half}
}
\usepackage[left=2cm,right=2cm,top=2cm,bottom=2cm]{geometry}

\theoremstyle{plain}
\newtheorem{theorem}{Theorem}
\newtheorem{lemma}[theorem]{Lemma}
\newtheorem{proposition}[theorem]{Proposition}
\newtheorem{corollary}[theorem]{Corollary}
\newtheorem{conjecture}[theorem]{Conjecture}
\newtheorem{example}[theorem]{Example}
\theoremstyle{definition}
\newtheorem{definition}[theorem]{Definition}
\newtheorem{assumption}[theorem]{Assumption}
\newtheorem{remark}[theorem]{Remark}

%% OBJECTS
\newcommand{\F}[1][n,p]{\mathbb{F}_{#1}}
\newcommand{\Ff}[1][n,p]{\overline{\mathbb{F}}_{#1}}
\newcommand{\N}{\mathbb{N}}
\newcommand{\R}{\mathbb{R}}
\newcommand{\B}{\mathbb{B}}

%% MORPHISMS
\newcommand{\Rep}[1][n,p]{\mathrm{i}_{#1}}
\newcommand{\Round}[1][n,p]{\mathrm{r}_{#1}}
\newcommand{\one}{\mathbbm{1}}
\newcommand{\id}{\mathrm{Id}}

%% DISTRIBUTIONS
\newcommand{\Unif}{\mathsf{Uniform}}

%% MATHS
\newcommand{\ceil}[1]{\lceil #1 \rceil}
\newcommand{\floor}[1]{\lfloor #1 \rfloor}
\newcommand{\intvl}[1]{\mathlarger{\left[\right.}  #1 \mathlarger{\left.\right]}}
\newcommand{\inv}{^{-1}}
\newcommand{\fintvl}[1][x]{\mathlarger{\lfloor}#1,#1\mathlarger{\rceil}}
\newcommand{\fp}{_{\mathrm{fp}}}
\newcommand{\absv}[1]{\vert #1\vert}
\newcommand{\Pro}[1]{\mathbb{P}\left[ #1 \right]}

\author{Fredrik Dahlqvist}
\title{Probabilistic accuracy and stability}

\begin{document}
\maketitle

\section{Probabilistic modelling}

\subsection{Notation}
Let $\F$ denote the set of floating-point numbers with exponent size $n$ and mantissa size $p$, let $\Round:\R\to\F$ and $\Rep: \F\to \R$ be the corresponding rounding and representation maps respectively. Since $\Round$ is monotone it follows that for each $n\in\F$ the set $\Round\inv(n):=\{x\in\R\mid \Round(x)=n\}$ is an interval (closed or open, it doesn't matter for what follows). In particular, given $x\in \R$ the set
\[
\Round\inv(\Round(x))=\{z\in\R\mid \Round(z)=\Round(x)\}
\]
is an interval. We define
\begin{enumerate}
\item $\ceil{x}=\sup \Round\inv(\Round(x))$
\item $\floor{x}=\inf \Round\inv(\Round(x))$
\end{enumerate}
For notational clarity we denote the interval $\left[\floor{x},\ceil{x}\right]$ as $\fintvl$.
Moreover, each interval $\fintvl$ contains a machine representable real which we denote by $\hat{x}$ and which is defined by
\[
\hat{x}:=\Rep(\Round(x))\in \fintvl
\]

\subsection{Modelling floating-point arithmetic probabilistically.}

Using the notation above we can define a Markov chain $u$ on $\R$ as follows
\[
u(x):=\Unif\left(\fintvl\right),
\]
the uniform distribution on $\fintvl$ whose pdf is given by $f(t)=\frac{1}{(\ceil{x}-\floor{x})}\one_{\fintvl}(t)$. We now model floating-point arithmetic operations. 

Given $\diamond\in \{+,-,\times,/\}$ we define $\widehat{\diamond}:\F\times \F\to \F$ as the machine implementation of $\diamond$, and we also define $\diamond\fp$ as the Markov chain on $\R\times \R$ given by
\[
\diamond\fp:=u\circ\diamond.
\]
For example:
\[
x+\fp y:=\Unif\fintvl[x+y]
\]
It follows from the IEEE 754 standard (in particular the `correct rounding requirement') that the machine implementation $\hat{\diamond}$ and the probabilistic model $\diamond\fp$ must be related related via:
\[
(\Round)_\ast(x\diamond\fp y)=\delta_{\Round(x)~\widehat{\diamond}~\Round(y)}
\]
or, alternatively for $x,y\in\F$
\[
\Rep(x)\diamond\fp \Rep(y)=\Unif\left(\fintvl[\Rep(x~\hat{\diamond}~y)]\right)
\]



\subsection{Probabilistic floating-point arithmetic on probabilistic inputs.}

Given two probability measure $\mu,\nu$ on the inputs of an arithmetic operation (i.e.\@ on $\R$), we can compute the probability of the output as the probability measure given by
\[
(\diamond\fp)_\ast(\mu\times\nu)=u_\ast\circ \diamond_\ast (\mu\times \nu)
\]
which assigns to an interval $\intvl{a,b}$ the probability
\begin{equation}\label{eq:noisyoperation}
\int_{-\infty}^{\infty} u(x)\left(\intvl{a,b}\right) ~d\diamond_\ast(\mu\times \nu)
\end{equation}
If we assume that $\mu,\nu$ are absolutely continuous w.r.t. the Lebesgue measure $\lambda$ we can find two density functions $\nicefrac{d\mu}{d\lambda}:=f_\mu,\nicefrac{d\nu}{d\lambda}:=f_\nu$ such that
\[
\mu(A)=\int_A f_\mu(x)~dx\qquad\text{and}\qquad\nu(A)=\int_A f_\nu(x)~dx
\]
The probability measure $\diamond_\ast(\mu\times \nu)$ is then absolutely continuous with respect to the Lebesgue measure and we can then compute its density (see \cite{springer1979algebra}) as:

\begin{align}
\frac{d(+)_\ast(\mu\times\nu)}{d\lambda}(t)&=\int_{-\infty}^{\infty} f_\mu(x)f_\nu(t-x)~dx\label{eq:pdfplus}\\
\frac{d(-)_\ast(\mu\times\nu)}{d\lambda}(t)&=\int_{-\infty}^{\infty} f_\mu(x)f_\nu(x-t)~dx\label{eq:pdfminus}\\
\frac{d(\times)_\ast(\mu\times\nu)}{d\lambda}(t)&=\int_{-\infty}^{\infty} \frac{1}{\absv{x}}f_\mu(x)f_\nu\left(\frac{t}{x}\right)dx\label{eq:pdftimes}\\
\frac{d(/)_\ast(\mu\times\nu)}{d\lambda}(t)&=\int_{-\infty}^{\infty} \absv{x}f_\mu(x)f_\nu(tx)dx\label{eq:pdfdiv}
\end{align}
From this we can compute \eqref{eq:noisyoperation}. Let $f$ be any of the densities above, \eqref{eq:noisyoperation} amounts to computing
\[
\int_a^b u(x)\left(\intvl{a,b}\right) f(x) ~dx
\]
Assume that $a,b$ are machine representable numbers and that $a\neq b$. Given $x\in \R$ there are four possibilities
\begin{enumerate}
\item $\fintvl \cap \intvl{a,b}=\emptyset$, i.e.\ $\ceil{x}<a$ or $b>\floor{x}$ in which case 
\[
u(x)\left(\intvl{a,b}\right)=0
\]
\item  $\floor{x}\leq a\leq \ceil{x}<b$ in which case
\[
u(x)\left(\intvl{a,b}\right)=\frac{\ceil{x}-a}{\ceil{x}-\floor{x}}
\]
\item $\fintvl \subseteq \intvl{a,b}$, i.e.\ $a<\floor{x}, \ceil{x}<b$ in which case
\[
u(x)\left(\intvl{a,b}\right)=1
\]
\item  $a< \floor{x}\leq b<\ceil{x}$ in which case
\[
u(x)\left(\intvl{a,b}\right)=\frac{b-\floor{x}}{\ceil{x}-\floor{x}}
\]
\end{enumerate}
Since we've assumed that $a<b$ are machine representable, as $x$ varies from $-\infty$ to $\infty$ each of the cases described above occur in the order 1.2.3.4.1. We can thus partition $\R$ into five intervals. 
\begin{enumerate}
\item On $\left(-\infty,\floor{a}\right)$ we have $\ceil{x}<a$ and thus
\[
\int_{-\infty}^{\floor{a}} u(x)\left(\intvl{a,b}\right) f(x) ~dx=0
\]
\item On $\fintvl[a]$ we have $\floor{x}=\floor{a}\leq a\leq \ceil{a}=\ceil{x}$ and thus
\[
\int_{\floor{a}}^{\ceil{a}} u(x)\left(\intvl{a,b}\right) f(x) ~dx=\frac{\ceil{a}-a}{\ceil{a}-\floor{a}}\int_{\floor{a}}^{\ceil{a}}  f(x) ~dx
\]
\item On $\left(\ceil{a},\floor{b}\right)$ we have  i.e.\ $a<\floor{x}, \ceil{x}<b$ and thus
\[
\int_{\ceil{a}}^{\floor{b}} u(x)\left(\intvl{a,b}\right) f(x) ~dx=\int_{\ceil{a}}^{\floor{b}} f(x)~dx
\]
(Note that if $b$ is the next machine representable number after $a$ we have $\left(\ceil{a},\floor{b}\right)=\emptyset$ and this case drops out.)
\item On $\fintvl[b]$ we have $\floor{x}=\floor{b}\leq b\leq \ceil{b}=\ceil{x}$ and thus
\[
\int_{\floor{b}}^{\ceil{b}} u(x)\left(\intvl{a,b}\right) f(x) ~dx=\frac{b-\floor{b}}{\ceil{b}-\floor{b}}\int_{\floor{b}}^{\ceil{b}}  f(x) ~dx
\]
\item Finally, on $\left(\ceil{b},\infty\right)$ we have
\[
\int_{\ceil{b}}^{\infty} u(x)\left(\intvl{a,b}\right) f(x) ~dx=0
\]
\end{enumerate}

\noindent It follows that when $a<b$ are machine representable we can write \eqref{eq:noisyoperation} as
\begin{align*}
\int_{-\infty}^{\infty} u(x)\left(\intvl{a,b}\right) ~d\diamond_\ast(\mu\times \nu)=\frac{\ceil{a}-a}{\ceil{a}-\floor{a}}\int_{\floor{a}}^{\ceil{a}}  f(x) ~dx + \int_{\ceil{a}}^{\floor{b}} f(x)~dx +
\frac{b-\floor{b}}{\ceil{b}-\floor{b}}\int_{\floor{b}}^{\ceil{b}}  f(x) ~dx
\end{align*}
We can also compute the pdf of the probability measure $(\diamond\fp)_\ast(\mu\times \nu)$ defined by \eqref{eq:noisyoperation}. We have
\begin{align*}
(\diamond\fp)_\ast(\mu\times \nu)((-\infty,t))&=\int_{-\infty}^\infty u(x)(-\infty,t))f(x) ~dx\\
&=\int_{-\infty}^\infty \int_{-\infty}^t \frac{1}{\ceil{x}-\floor{x}} \one_{\fintvl}(y) f(x) ~dy dx\\
&=\int_{-\infty}^t\int_{-\infty}^\infty\frac{1}{\ceil{x}-\floor{x}} \one_{\fintvl}(y) f(x) ~dx dy
\end{align*}
where the last step is imply an application of Fubini's theorem. It follows that
\begin{align*}
\frac{d}{dt}(\diamond\fp)_\ast(\mu\times \nu)((-\infty,t))&=\int_{-\infty}^\infty\frac{1}{\ceil{x}-\floor{x}} \one_{\fintvl}(t) f(x) ~dx\\
&=\int_{\floor{t}}^{\ceil{t}}\frac{f(x)}{\ceil{t}-\floor{t}} ~dx
\end{align*}
This holds for \emph{any} initial density, in other words if $\mu$ is a probability measure with Lebesgue density $\nicefrac{d\mu}{d\lambda}=f$, then $u_\ast(\mu)$ has density
\begin{equation}
\frac{du_\ast(\mu)}{d\lambda}(t)=\int_{\floor{t}}^{\ceil{t}}\frac{f(x)}{\ceil{t}-\floor{t}} ~dx \label{eq:updf}
\end{equation}

\subsection{The quantization operator $Q$}

\newcommand{\Mes}{\mathrm{M}}

The Markov chain $u$ defines a linear operator acting on measures on $\R$ via 
\[
Q: \Mes(\R)\to\Mes(\R), \mu\mapsto u_\ast(\mu)
\]
If a measure $\mu$ on $\R$ is absolutely continuous w.r.t. to the Lebesgue measure $\lambda$, then the operator $Q$ transform its density according to \eqref{eq:updf}, i.e.
\[
\frac{dQ(\mu)}{d\lambda}(t)=\int_{\floor{t}}^{\ceil{t}} \frac{1}{\ceil{t}-\floor{t}} \frac{d\mu}{d\lambda}~d\lambda 
\] 
We note immediately that $\frac{dQ(\mu)}{d\lambda}$ is piecewise constant, in fact constant over each interval $\fintvl[\Rep(i)], i\in \F$, which is why we will call $Q$ the $\F$-quantization operator, or simply the quantization operator if there is no ambiguity about $n,p$.

\subsection{Arithmetic of quantized distributions}

We compute \eqref{eq:pdfplus}-\eqref{eq:pdfdiv} for pdfs. It is clear from \eqref{eq:updf} that these are piecewise constant, i.e. a quantized pdf can always be expressed as a weighted sum of indicator functions:
\[
\sum_{i=1}^N \alpha_i \one_{\left[a_i,b_i\right]}
\]
Let us therefore start by computing \eqref{eq:pdfplus}-\eqref{eq:pdfdiv} for (scaled) indicator functions.

\subsubsection{Sums}
We denote by $\oplus$ the convolution of \eqref{eq:pdfplus} and we first compute the convolution of two weighted indicators:
\begin{align*}
\alpha\one_{[a,b]}\oplus\beta\one_{[c,d]}(t)& =\int_{-\infty}^\infty\alpha\one_{[a,b]}(x)\beta\one_{[c,d]}(t-x)dx \\
&=\alpha\beta\int_a^b \one_{[c,d]}(t-x)~dx\\
&= \begin{cases}
\alpha\beta(b-a)&c+b\leq t\text{ and }t\leq d+a\\
\alpha\beta((d+b)-t) & c+b\leq t\text{ and } d+a\leq t \\
\alpha\beta(t-(a+c)) & t\leq c+b\text{ and }t\leq d+a\\
\alpha\beta(d-c) & t\leq c+b\text{ and } d+a\leq t\\
0 &\text{else}
\end{cases}
\end{align*}

Note that convolutions are commutative and it follows by linearity of integration that
\begin{align*}
\sum_{i=1}^N \alpha_i \one_{\left[a_i,b_i\right]} \oplus \sum_{i=1}^N \alpha_i \one_{\left[a_i,b_i\right]}(t)&=\sum_{i,j=1}^N \alpha_i\one_{[a_1,b_1]}\oplus\beta_j\one_{[c_j,d_j]}(t) \\
&= 2\sum_{i\leq j=1}^N \alpha_i\one_{[a_1,b_1]}\oplus\beta_j\one_{[c_j,d_j]}(t)
\end{align*}


\subsubsection{Differences}

In the same way as above we compute 
\begin{align*}
\alpha\one_{[a,b]}\ominus\beta\one_{[c,d]}(t)&=\int_{-\infty}^{\infty} \alpha\one_{[a,b]}(x)\beta\one_{[c,d]}(x-t)~dx\\
&=\alpha\beta\int_a^b \one_{[c,d]}(x-t)~dx\\
&=\begin{cases}
\alpha\beta(d-c)&\text{if }a-c\leq t\text{ and }t\leq b-d\\
\alpha\beta((b-c)-t) &\text{if }a-c\leq t\text{ and }b-d\leq t\\
\alpha\beta(t-(a-d)) & \text{if }t\leq a-c\text{ and }t\leq b-d\\
\alpha\beta(b-a) & \text{if } t\leq a-c\text{ and }b-d\leq t\\
0&\text{else}
\end{cases}
\end{align*}
Note that, just like substraction, $\ominus$ is not commutative.

\subsection{Multiplication}
For notational clarity we assume $\alpha=\beta=1$
\begin{align*}
\int_{-\infty}^\infty \frac{1}{\absv{x}}\one_{[a,b]}(x)\one_{[c,d]}\left(\frac{t}{x}\right)~dx = &\int_{a}^b \frac{1}{\absv{x}}\one_{[c,d]}\left(\frac{t}{x}\right)~dx\\
=& \int_{\min(a,0)}^{\min(b,0)}\frac{1}{\absv{x}}\one_{[c,d]}\left(\frac{t}{x}\right)~dx+\int_{\max(a,0)}^{\max(b,0)}\frac{1}{\absv{x}}\one_{[c,d]}\left(\frac{t}{x}\right)~dx\\
=& \int_{\min(a,0)}^{\min(b,0)}\frac{1}{\absv{x}}\one_{[\min(c,0),\min(d,0)]}\left(\frac{t}{x}\right)~dx+\int_{\min(a,0)}^{\min(b,0)}\frac{1}{\absv{x}}\one_{[\max(c,0),\max(d,0)]}\left(\frac{t}{x}\right)~dx \\
& +\int_{\max(a,0)}^{\max(b,0)}\frac{1}{\absv{x}}\one_{[\min(c,0),\min(d,0)]}\left(\frac{t}{x}\right)~dx+\int_{\max(a,0)}^{\max(b,0)}\frac{1}{\absv{x}}\one_{[\max(c,0),\max(d,0)]}\left(\frac{t}{x}\right)~dx\\
:= & A+B+C+D
\end{align*}

\paragraph{Term D:}
\begin{align*}
D& =\int_{\max(a,0)}^{\max(b,0)}\frac{1}{\absv{x}}\one_{[\max(c,0),\max(d,0)]}\left(\frac{t}{x}\right)~dx\\
&=\begin{cases}
\int_{\frac{t}{\max(d,0)}}^{\frac{t}{\max(c,d)}}\frac{1}{x}~dx=\log\max(d,0)-\log\max(c,0)&\text{if }\max(a,0)\leq \frac{t}{\max(d,0)}\leq \frac{t}{\max(c,0)}\leq \max(b,0)\\
0&\text{else}
\end{cases}
\end{align*}

\section{Computing instability probabilities}

We now show how to compute some simple instability probabilities.

\subsection{Programs with no operation}


\begin{lstlisting}
/* Precondition: Input is distributed uniformly on [0,1] */

half x = ReadInput();
if (x>0.25){
  return TRUE;}
else{
  return FALSE;} 
\end{lstlisting}

Let us call the program above $\tt Prog$ and assume that the input is a real number drawn randomly and uniformly from $\intvl{0,1}$. The function \texttt{ReadInput} rounds the input to a floating-point number in $\F$. The variable $\tt x$ is thus distributed according to
\[
(\Round)_\ast(\Unif(0,1))
\]
For $X\sim\Unif(0,1)$ we are interested in computing the probability of misclassification, i.e.\@
\[
\Pro{\mathtt{Prog}(X)=\mathtt{FALSE}\mid X> 0.25}
\]
This can be computed explicitly as
\begin{align*}
\Pro{\mathtt{Prog}(X)=\mathtt{FALSE}\mid X\leq 0.25}&=\frac{\Pro{\mathtt{Prog}(X)=\mathtt{FALSE} \wedge X\leq 0.25}}{\Pro{X> 0.25}}\\
&=\frac{\Pro{X\in \{x\mid x>0.25 \wedge \Round[5,10](x)\leq 0.25\}}}{0.75}\\
&=\frac{\ceil{0.25}-0.25}{0.75}
\end{align*}

More generally if $X$ is distributed according to some density $f_X$ and $C$ is the machine-representable value of the threshold we have
\[
\Pro{\mathtt{Prog}(X)=\mathtt{FALSE}\mid X> C}=\frac{\int_C^{\ceil{C}} f_X(x)~dx}{\int_{x>C} f_X(x)~dx}
\]

%\[
%\xymatrix@C=12em
%{
%\R\ar[r]^{\Round[5,10]} & \F[5,10]\ar[r]^{->0.25} & \B
%}
%\]

\subsubsection*{b) Programs with one operation}

\newcommand{\rRound}{\mathrm{Round}}

\begin{lstlisting}
/* Precondition: Inputs are distributed uniformly on [0,1] */

half x = ReadInput();
half y = ReadInput();
if (x+y>0.25){
  return TRUE;}
else{
  return FALSE;} 
\end{lstlisting}

This time we want to compute:
\[
\Pro{\mathtt{Prog}(X,Y)=\mathtt{FALSE}\mid X+Y> C}.
\]
This is more complex because we model $\mathtt{Prog}(X,Y)$ by using $+\fp$. For notational clarity we define $\rRound:=\Rep\circ \Round$.
\begin{align*}
&\Pro{\mathtt{Prog}(X,Y)=\mathtt{FALSE}\mid X+Y> C}\\
:=&\Pro{(u\circ + \circ (\rRound\times \rRound))(X,Y)\in\{(-\infty,C]\}\mid X+Y>C}\\
=&\int_{C}^{\infty}\left( \int_{\R^2}(u\circ + \circ (\rRound\times \rRound))(x,y)(-\infty,C]
~\frac{\phi_\mu(x)\phi_\nu(y)}{\phi_\mu\ast\phi_\nu(z)}\delta(z-x-y) ~dx~dy\right)\phi_\mu\ast\phi_\nu(z)~dz\\
=&\int_{C}^{\infty} \int_{-\infty}^\infty (u\circ + \circ (\rRound\times \rRound))(x,z-x)(-\infty,C]~\phi_\mu(x)\phi_\nu(z-x)~dx~dz\\
=&\int_{C}^{\infty} \int_{-\infty}^\infty u(\rRound(x)+\rRound(z-x))(-\infty,C]\phi_\mu(x)\phi_\nu(z-x)~dx~dz 
\end{align*}
%Note that:
%\begin{itemize}
%\item $u(t)(-\infty,C]=0$ if $t>\ceil{C}$. Moreover, 
%\item $u(t)(-\infty,C]=\frac{C-\floor{C}}{\ceil{C}-\floor{C}}$ if $\floor{C}\leq t\leq \ceil{C}$
%\item $u(t)(-\infty,C]=1$ if $t<\floor{C}$
%\item $C\leq z$
%\end{itemize} 
%Note also that
%\begin{lemma}\label{lem:distrib}
%The following equivalences hold:
%\begin{enumerate}
%\item $\rRound(x)+\rRound(z-x)>\ceil{C}\Leftrightarrow z>\ceil{C}$
%\item $\rRound(x)+\rRound(z-x)<\floor{C}\Leftrightarrow z<\floor{C}$
%\end{enumerate}
%\end{lemma}
%\begin{proof}
%We have:
%\begin{align*}
%&\rRound(x)+\rRound(z-x)>\ceil{C}\\
%\Leftrightarrow & \floor{\rRound(x)+\rRound(z-x)}>\ceil{C}\\
%\Leftrightarrow &\floor{\rRound(\rRound(x)+\rRound(z-x))}>\ceil{C}\\
%\Leftrightarrow & (1-u)(\rRound(x)+\rRound(z-x))>\ceil{C}\\
%\Leftrightarrow & (1-u)\rRound(x)+(1-u)\rRound(z-x)>\ceil{C}\\
%\Leftrightarrow & \floor{\rRound(x)}+\floor{\rRound(z-x)}>\ceil{C}\\
%\Leftrightarrow & \floor{x}+\floor{z-x}>\ceil{C}\\
%\Leftrightarrow & (1-u)x+(1-u)(z-x)>\ceil{C}\\
%\Leftrightarrow &\floor{z}>\ceil{C}\\
%\Leftrightarrow & z>\ceil{C}
%\end{align*}
%and similarly for the second equivalence.
%\end{proof}
%
%From the first equivalence we get that $u(\rRound(x)+\rRound(z-x))(-\infty,C]=0$ when  $z>\ceil{C}$ and from the second equivalence we see that $u(\rRound(x)+\rRound(z-x))(-\infty,C]=1$ if $z<\floor{C}$, which contradicts $C\leq z$, i.e. this case does not present itself. Thus the only non-zero contribution is when 
%\[
%\floor{C}<\rRound(z)<\ceil{C}
%\]
%The integral above now become
%\begin{align*}
%\Pro{\mathtt{Prog}(X,Y)=\mathtt{FALSE}\mid X+Y> C}=&\int_C^{\ceil{C}}\int_{-\infty}^\infty \frac{C-\floor{C}}{\ceil{C}-\floor{C}}\phi_\mu(x)\phi_\nu(z-x)~dx~dz \\
%=&\int_C^{\ceil{C}}\frac{C-\floor{C}}{\ceil{C}-\floor{C}}~\phi_\mu\ast\phi_\nu(z)~dz
%\end{align*}
%
%\begin{remark}
%Lemma \ref{lem:distrib} does \textbf{not} hold for multiplication and division since it relies on the distributivity of $+$ over $\times$. It also does \textbf{not} hold for fixed point representation since it also relies on expressing $\floor{x}$ and $(1-u)x$. 
%\end{remark}
\subsection*{More than one operation}

Consider now a program of the shape
\begin{lstlisting}
/* Precondition: Inputs are distributed uniformly on [0,1] */

half x = ReadInput();
half y = ReadInput();
half z = ReadInput();
if (1+x+(y*z)/2>0.25){
  return TRUE;}
else{
  return FALSE;} 
\end{lstlisting}

The function $\mathtt{1+x+(y*z)/2}$ provides an over-approximation of the distribution of $\mathtt{1+x+(x*x)/2}$, a common approximation of $\mathtt{exp(x)}$. We can consider (1) the distribution of $\mathtt{1+x+(y*z)/2}$ and (2) the probability of misclassification.

Let's start with computing the distribution of $\mathtt{1+x+(y*z)/2}$. We define $\upsilon$ to be the uniform measure on $\intvl{0,1}$. The measure we're looking for is given by the pushforward of the measure $\upsilon\times\upsilon\times\upsilon$ through the sequence of (probabilistic) maps
\[
\xymatrix@C=12ex
{
\R^3\ar[r]^{\langle \id, \times\rangle} & \R^2\ar[r]^{\langle\id,\frac{-}{2}\rangle} & \R^2\ar[r]^{\langle\id,u\rangle} & \R^2 \ar[r]^{+} & \R\ar[r]^{-+1} &\R\ar[r]^{u}&\R
}
\]
Here we assume for simplicity's sake that we can do $(\cdot\times\cdot)/2$ and $1+\cdot+\cdot$ in one operation. We know that the pdf of $((\cdot \times \cdot)/2)_\ast(\upsilon\times\upsilon)$ is given via \eqref{eq:pdftimes} by
\begin{align*}
\phi(t)&=\frac{1}{2}\int_{-\infty}^\infty \frac{1}{\absv{x}} \one_{\intvl{0,1}}(x) \one_{\intvl{0,1}}\left(\frac{t}{x}\right)~dx\\
&=\frac{1}{2}\int_0^\infty \frac{1}{x}\one_{\intvl{0,1}}(x) \one_{\intvl{0,1}}\left(\frac{t}{x}\right)~dx+\int_{-\infty}^0 -\frac{1}{x}\one_{\intvl{0,1}}(x) \one_{\intvl{0,1}}\left(\frac{t}{x}\right)~dx\\
&=\frac{1}{2}\int_0^1 \frac{1}{x}\one_{\intvl{0,1}}\left(\frac{t}{x}\right)~dx\\
&=\frac{1}{2}\one_{\intvl{0,1}}(t)\int_t^1 \frac{1}{x}~dx \\
&=\frac{1}{2}\one_{\intvl{0,1}}(t) (-\log(t))
\end{align*}
And thus thus, the pdf of $(u\circ (\cdot \times \cdot)/2))_\ast(\upsilon\times\upsilon)$ is given via \eqref{eq:updf} by
\begin{align*}
\hat{\phi}(t)&=\sum_{i\in \Rep \left[\F\right]} \one_{\intvl{\floor{i},\ceil{i}}}(t) \int_{\floor{i}}^{\ceil{i}}\frac{\phi(x)}{\ceil{i}-\floor{i}} ~dx\\
&=\sum_{i\in \Rep \left[\F\right]\cap\intvl{0,1}}  \one_{\intvl{\floor{i},\ceil{i}}}(t) \frac{\ceil{i}(1-\log(\ceil{i})-\floor{i}(1-\log(\floor{i})}{\ceil{i}-\floor{i}}
\end{align*}
We can now compute the density of $(1+\cdot+\cdot)\ast$ applied to the product of $(u\circ (\cdot \times \cdot)/2))_\ast(\upsilon\times\upsilon)$  and $\upsilon$ via \eqref{eq:pdftimes}:
\begin{align*}
\psi(t)=\int_{-\infty}^\infty \hat{\phi}(x) \one_{\intvl{1,2}}(t-x)~dx
\end{align*}
\bibliographystyle{plain}
\bibliography{george} 
\end{document}